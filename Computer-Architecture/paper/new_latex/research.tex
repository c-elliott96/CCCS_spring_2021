\documentclass[12pt,letterpaper]{article}
\usepackage{ifpdf}
\usepackage{mla}
\usepackage[american]{babel}
\usepackage{csquotes}
\usepackage[style=mla,backend=biber]{biblatex}

%% \begin{filecontents}{citations.bib}
%%      @article{Barrett2009,
%%      Author = {Barrett, Scott},
%%      Title = {The Coming Global Climate-Technology Revolution},
%%      Journal = {Journal of Economic Perspectives},
%%      Volume = {23},
%%      Number = {2},
%%      Year = {2009},
%%      Month = {June},
%%      Pages = {53-75},
%%      DOI = {10.1257/jep.23.2.53},
%%      URL = {https://www.aeaweb.org/articles?id=10.1257/jep.23.2.53}}
%% \end{filecontents}

\addbibresource{citations.bib}

\begin{document}
%%\begin{mla}{Christian}{Elliott}{Manglik}{CISS-420}{\today}{Climate Change and its Relationship with Technology}
  From the 17th century through modern day, the world has plunged head-first
into the era of technological development and its companion, capitalism. For
the last 300 years we have scaled and progressed continuously in our quest
for innovation and power. At every discovery, technology has stood in lock-step
with us, acting as our platform upon which we were able to continuously climb
the ladder as the most dominant and intelligent species our planet has to offer.
But we have done so recklessly. We don't often enough ask ourselves, as a
species, if we are behaving in a way that will preserve the conditions in which
humanity can continue to exist. Because we haven't been emphasizing this
concept, humanity has for decades now treated this world as its eternal supplier
of life, and all things we could ever consume.

It seems, though, that even mother Earth has her limits. The global temperature
is increasing year-over-year, sea levels are rising, and polar ice has reached
record lows. These concerns are alarming in their own right, but when considered
as part of a larger perspective, we realize that our planet's system is balanced
in a way that has allowed life to exist for a long time. It exists in a
symbiotic state of a network of processes and mechanisms that make our planet
/work/. But when an element of such a system (that is we) begins to erode the
foundation upon which such a massively complex system sits, we will find that
it might not take long for the system to begin to lose integrity. More worrisome
still, we don't know very well how quickly such a super-system might collapse.
It could take centuries, perhaps. Or a few decades. Or 4 years. We cannot
predict the ultimate failure points.

There is, however, technology. The same human-ness that got us into such
existential circumstances is the same mechanism by which we can avoid a
premature demise. In the coming years and decades, humanity will be forced to
quickly invest in global operations to mitigate the natural effects of climate
change. These endeavors will likely involve extensive research in fields like
Artificial Intelligence, quantum mechanics, energy production and distribution,
and food production. Text \cite[55]{Barrett2009}

Text

\printbibliography
%%\end{mla}
\end{document}

%% #     [CITATION] IPCC, 2018: Summary for Policymakers. In: Global Warming of 1.5°C.
%% #     An IPCC Special Report on the impacts of global warming of 1.5°C above
%% #     pre-industrial levels and related global greenhouse gas emission pathways,
%% #     in the context of strengthening the global response to the threat of climate
%% #     change, sustainable development, and efforts to eradicate poverty
%% #     [Masson-Delmotte, V., P. Zhai, H.-O. Pörtner, D. Roberts, J. Skea, P.R. Shukla,
%% #     A. Pirani, W. Moufouma-Okia, C. Péan, R. Pidcock, S. Connors, J.B.R. Matthews,
%% #     Y. Chen, X. Zhou, M.I. Gomis, E. Lonnoy, T. Maycock, M. Tignor, and
%% #     T. Waterfield (eds.)]. World Meteorological Organization, Geneva,
%% #     Switzerland, 32 pp.

%% #     [A.1] Human activities are estimated to have caused approximately 1.0 degrees C of
%% #     global warming above pre-industrial levels, with a /likely/ range of 0.8
%% #     degrees C to 1.2 degrees C. Global warming is /likely/ to reach 1.5 degrees C
%% #     between 2030 and 2052 if it continues to increase at the current rate.

%% #     [A.1.3] Trends in intensity and frequency of some climate and weather extremes
%% #     have been detected over time spans during which about 0.5 degrees C of
%% #     global warming occurred (medium confidence). This assessment is based on
%% #     several lines of evidence, inccluding attribution studies for changes in
%% #     extremes since 1950.

%% #     [B.1] Climate models project robust differences in regional climate
%% #     characteristics between present-day and global warming of 1.5 degrees C, and
%% #     between 1.5 degrees C and 2 degrees C. These differences include increases in:
%% #     mean temperature in most land and ocean regions (high confidence), hot
%% #     extremes in most inhabited regions (high confidence), heavy precipitation in
%% #     several regions (medium confidence), and the probability of drought and
%% #     precipitation deficits in some regions (medium confidence).
   
%% #     [B.5] Climate-related risks to health, livelihoods, food security, water
%% #     supply, human security, and economic growth are projected to increase with
%% #     global warming of 1.5 degrees C and increase further with 2 degrees C.

%% #     [B.5.3] Limiting warming to 1.5°C compared with 2°C is projected to result in
%% #     smaller net reductions in yields of maize, rice, wheat, and potentially other
%% #     cereal crops, particularly in sub-Saharan Africa, Southeast Asia, and Central
%% #     and South America, and in the CO2-dependent nutritional quality of rice and
%% #     wheat (high confidence). Reductions in projected food availability are larger
%% #     at 2°C than at 1.5°C of global warming in the Sahel, southern Africa, the
%% #     Mediterranean, central Europe, and the Amazon (medium confidence). Livestock
%% #     are projected to be adversely affected with rising temperatures, depending
%% #     on the extent of changes in feed quality, spread of diseases, and water
%% #     resource availability (high confidence).

%% #     [C.2] Pathways limiting global warming to 1.5°C with no or limited
%% #     overshoot would require rapid and far-reaching transitions in energy, land,
%% #     urban and infrastructure (including transport and buildings), and industrial
%% #     systems (high confidence). These systems transitions are unprecedented in
%% #     terms of scale, but not necessarily in terms of speed, and imply deep
%% #     emissions reductions in all sectors, a wide portfolio of mitigation options
%% #     and a significant upscaling of investments in those options (medium
%% #     confidence).

%% #     [CITATION (BibTex)] 
%% #     Close
%% #     @article{10.1257/jep.23.2.53,
%% #     Author = {Barrett, Scott},
%% #     Title = {The Coming Global Climate-Technology Revolution},
%% #     Journal = {Journal of Economic Perspectives},
%% #     Volume = {23},
%% #     Number = {2},
%% #     Year = {2009},
%% #     Month = {June},
%% #     Pages = {53-75},
%% #     DOI = {10.1257/jep.23.2.53},
%% #     URL = {https://www.aeaweb.org/articles?id=10.1257/jep.23.2.53}}

%% #     (54) Emissions of CO$_2$ and other greenhouse gases can be reduced
%% #     significantly using existing technologies, but stabilizing concentrations
%% #     will require a technological revolution - a "revolution" because it will
%% #     require fundamental change, achieved within a relatively short period of
%% #     time.

%% #     (58) “Concentrated solar power” is a technology that, as the name implies,
%% #     raises the density of solar energy using mirrors to produce heat, which can
%% #     then be used to turn a turbine for electricity generation. An example is the
%% #     “power tower,” a system of sun-tracking mirrors that beam concentrated solar
%% #     power to a receiver at the top of a tower, through which flows a working
%% #     liquid for driving the turbine. An advantage of this technology is that it
%% #     can be scaled to the size of a central power plant (individual units are being
%% #     designed to generate up to 250 megawatts of power). It can also store thermal
%% #     energy to produce electricity at night, addressing the problem of
%% #     intermittency.

%% #     (59) A more radical idea is “space solar power.” This technology would use
%% #     huge photovoltaic arrays to capture the sun’s energy in space, convert it to
%% #     direct electrical current, and then beam the electricity to Earth using
%% #     microwaves or lasers. To produce this energy, solar satellites would be
%% #     placed in high altitude, geosynchronous orbit, and spaced far enough apart
%% #     so that at least one unit faced the sun at all times—a solution to the
%% #     intermittency problem.

%% #     (59) An expansion of nuclear energy has the potential to reduce greenhouse
%% #     gas emissions significantly and within decades using proven technology. It
%% #     also has disadvantages. Addressing these will require innovation and
%% #     institutional changes.
