\documentclass[a4paper,
               headsepline,footsepline,
               openany,oneside,chapterprefix,
               12pt]
{scrartcl}

\usepackage{myassignment}

\renewcommand\AUTHOR{Dr.~Yihsiang Liow}
\renewcommand\SHORTAUTHOR{Dr.~Y.~Liow}

\renewcommand\COURSENAME{Cryptography and Computer Security}
\renewcommand\COURSESHORTNAME{Crypto \& Security}
\renewcommand\COURSENUMBER{CISS451}


\renewcommand\TITLE{Final exam (takehome)}
\newcommand\LISTMARGIN{1in}
\renewcommand{\thethm}{\arabic{thm}}


\begin{document}
\topmatter

\newpage
\textsc{Typesetting aligned equations with comments/justifications}

Here's an example of typesetting aligned computations (with justifications).
Suppose I want to prove
$(x + y \cdot z) + (-y) \cdot z) = x$.
And I can only use the following:

Let $(R, +, \cdot, 0, 1)$ be a ring. 
\begin{enumerate}[nosep]
 \li The definition of $R$ being a ring, i.e., the ring axioms of $R$.
 \li Fact 1: If $x \in R$, then $0 \cdot x = 0$.
 \li Fact 2: If $x \in R$, then $0 \cdot x = 0 = x \cdot 0$.
 \li Fact 3: If $x + y = 0$, then $y = -x$.
 \li Fact 4: $y + x = 0$, then $y = -x$.
 \li Fact 5: If $x \in R$, then $-(-x) = x$.
\end{enumerate}

The I will show $(x + y \cdot z) + (-y) \cdot z) = x$
like this:

\begin{align*}
(x + y\cdot z) + (-y) \cdot z &= x + (y \cdot z + (-y) \cdot z) & & \text{by the associativity axiom of $+$} \\
                              &= x + (y + (-y)) \cdot z         & & \text{by the distributivity axiom} \\
                              &= x + 0 \cdot z                  & & \text{by the inverse axiom of $+$} \\
                              &= x + 0                          & & \text{by Fact 2} \\
                              &= x                              & & \text{by neutrality axiom of $+$} \\
\end{align*}

Take a look at the \LaTeX\ code.
The \verb!&! are alignment characters.


%===============================================================================
\newpage
Q1.
What is the ones digit of the following number
\[
1357^{{2468}^{{3579}^{{4680^{5791^{{6802^{7913^{8024^{9135}}}}}}}}}}
\]
A complete proof is required.

\textsc{Solution}


(a) Goal: Prove that id is a group isomorphism

\begin{enumerate}[nosep]
\item[] $\forall x \in G, id(x) = x$, thus id is reflexive by definition of id \\
\item[] $id_G(x) = x = id_G(x)$ thus, id is symmetric \\
\item[] $id_G(x) = x' = x$ so $id_G(x') = x'' = x$ and thus id is transitive \\
\end{enumerate}

(b)
\begin{enumerate}[nosep]
\item[] If $x,y \in G'$, then $ f(a)= x$, $f(b) = y$ for some $a,b \in G$
\item[] thus \begin{align*}
  f^{-1}(x * y) &= f^{-1}(f(a) * f(b)) \\
  &= f^{-1}(x) * f^{-1}(y) \\
  &= f^{-1}(x * y) \\
\end{align*}
\end{enumerate}

(c)
\begin{enumerate}[nosep]
\item[] Let $x \in G''$. Then $f'(a) = x$ for some $a \in G'$. 
\item[] Let $a \in G'$. Then $f(b) = a$ for some $b \in G$.
\item[] Since $a \in G'$ is attained through $b \in G$, and $x \in G''$ is attained through
  $a \in G'$, we have shown that function composition is a group isomorphism.
\end{enumerate}


%===============================================================================
\newpage
Q2.
In this question, you will prove several basic facts about groups.

In the proofs below, you assume use the following:
Let $(G, *, e)$ be a ring. 
\begin{enumerate}[nosep]
 \li The definition of $G$ being a group, i.e., the group axioms of $R$.
 \li Fact 1: Identity element is unique. In other words let $e,e' \in G$
 such that
 \begin{align*} 
  e * x &= x = x * e \\
  e' * x &= x = x * e'   
 \end{align*}  
 for all $x \in G$. Then $e = e'$.
 (This is proposition 202.2.1 in the notes.)
 \li Fact 2: Inverse of an element is unique.
 In other words let $x \in G$. Suppose $y,y' \in G$ such that
 \begin{align*} 
  x * y &= e = y * x \\
  x * y' &= e = y' * x   
 \end{align*}
 Then $y = y'$.
 (This is proposition 202.2.2 in the notes.)
 \li Fact 3: Left cancellation holds.
 In other words,
 let $a,x,y \in G$ such that
 \[
 a * x = a * y
 \]
 then
 \[
 x = y
 \]
 Likewise, if
 \[
 x * a = y * a
 \]
 then
 \[
 x = y
 \]
\end{enumerate}
Do not use any justification other than the axioms and Facts 1-3.

Prove the following
\begin{enumerate}
\item[(a)] $(x^{-1})^{-1} = x$.
\item[(b)] $(x * y)^{-1} = y^{-1} * x^{-1}$.
\end{enumerate}

\textsc{Solution}


(a) Goal: Prove that id is a group isomorphism

\begin{enumerate}[nosep]
\item[] $\forall x \in G, id(x) = x$, thus id is reflexive by definition of id \\
\item[] $id_G(x) = x = id_G(x)$ thus, id is symmetric \\
\item[] $id_G(x) = x' = x$ so $id_G(x') = x'' = x$ and thus id is transitive \\
\end{enumerate}

(b)
\begin{enumerate}[nosep]
\item[] If $x,y \in G'$, then $ f(a)= x$, $f(b) = y$ for some $a,b \in G$
\item[] thus \begin{align*}
  f^{-1}(x * y) &= f^{-1}(f(a) * f(b)) \\
  &= f^{-1}(x) * f^{-1}(y) \\
  &= f^{-1}(x * y) \\
\end{align*}
\end{enumerate}

(c)
\begin{enumerate}[nosep]
\item[] Let $x \in G''$. Then $f'(a) = x$ for some $a \in G'$. 
\item[] Let $a \in G'$. Then $f(b) = a$ for some $b \in G$.
\item[] Since $a \in G'$ is attained through $b \in G$, and $x \in G''$ is attained through
  $a \in G'$, we have shown that function composition is a group isomorphism.
\end{enumerate}


%===============================================================================
\newpage
Q3.
Assume the given facts about groups from Q2.
Furthermore define $x^n$ for $n \geq 0$ as follows:
\[
x^n =
\begin{cases}
e           & \text{ if $n = 0$ } \\
x           & \text{ if $n = 1$ } \\
x^{n-1} * x  & \text{ if $n > 1$ }
\end{cases}
\]
This is from the notes which also contains the definition of $x$ raised to a
negative power.

Prove that
\[
(x^n)^{-1} = (x^{-1})^n
\]
for $n \geq 0$ by induction.
(This above is also true when $n$ is negative. But you need to prove
the above for negative $n$.)


%==============================================================================
\newpage
Q4. Consider the ring $R = (\Z/2)[X]/n$ where $n = X^2 + 1$.
\begin{enumerate}[nosep]

 \item
 What is $|R|$, the size of $R$?

 \item
 Factorize $n = X^2 + 1$ in $(\Z/2)[X]$.

 \item
 For each element $x$ of $R$, write down the multiplicative inverse of $x$.
 Is $R$ a field?
 
\end{enumerate}

\textsc{Solution}


(a) Goal: Prove that id is a group isomorphism

\begin{enumerate}[nosep]
\item[] $\forall x \in G, id(x) = x$, thus id is reflexive by definition of id \\
\item[] $id_G(x) = x = id_G(x)$ thus, id is symmetric \\
\item[] $id_G(x) = x' = x$ so $id_G(x') = x'' = x$ and thus id is transitive \\
\end{enumerate}

(b)
\begin{enumerate}[nosep]
\item[] If $x,y \in G'$, then $ f(a)= x$, $f(b) = y$ for some $a,b \in G$
\item[] thus \begin{align*}
  f^{-1}(x * y) &= f^{-1}(f(a) * f(b)) \\
  &= f^{-1}(x) * f^{-1}(y) \\
  &= f^{-1}(x * y) \\
\end{align*}
\end{enumerate}

(c)
\begin{enumerate}[nosep]
\item[] Let $x \in G''$. Then $f'(a) = x$ for some $a \in G'$. 
\item[] Let $a \in G'$. Then $f(b) = a$ for some $b \in G$.
\item[] Since $a \in G'$ is attained through $b \in G$, and $x \in G''$ is attained through
  $a \in G'$, we have shown that function composition is a group isomorphism.
\end{enumerate}


%==============================================================================
\newpage
Q5.
Can RSA be extended to three primes?
In other words let $p,q,r$ be \textit{three} (not two) primes and let $N = pqr$.
$\phi(N)$ is the Euler totient of $N$.
Let $e,d$ are integers such that $ed \equiv 1 \pmod{N}$.
Let $x$ be an integer.
Then
\[
\left( x^e \right)^d \equiv x \pmod{N} \tag{*}
\]
If the above is not true, provide a counter-example.
Otherwise prove $(*)$.

\textsc{Solution}

(I'll do the first few steps for you.)

Let $G = \{e, A, B, C, D, E\}$ be a group of size 6. (Therefore
all the symbols $e, A, B, C, D, E$ are distinct.)
where $e$ is the identity element.
We have to complete the following group table:
\begin{longtable}{|c||c|c|c|c|c|c|}
\hline
$*$ & $e$ & $A$ & $B$ & $C$ & $D$ & $E$ \\ \hline\hline
$e$ &     &     &     &     &     &     \\ \hline
$A$ &     &     &     &     &     &     \\ \hline
$B$ &     &     &     &     &     &     \\ \hline
$C$ &     &     &     &     &     &     \\ \hline
$D$ &     &     &     &     &     &     \\ \hline
$E$ &     &     &     &     &     &     \\ \hline
\end{longtable}
Since $e$ is the identity element we have
\begin{longtable}{|c||c|c|c|c|c|c|}
\hline
$*$ & $e$ & $A$ & $B$ & $C$ & $D$ & $E$ \\ \hline\hline
$e$ & $e$ & $A$ & $B$ & $C$ & $D$ & $E$ \\ \hline
$A$ & $A$ &     &     &     &     &     \\ \hline
$B$ & $B$ &     &     &     &     &     \\ \hline
$C$ & $C$ &     &     &     &     &     \\ \hline
$D$ & $D$ &     &     &     &     &     \\ \hline
$E$ & $E$ &     &     &     &     &     \\ \hline
\end{longtable}
$AA$ can be $e, B, C, D$, or $E$.

\textsc{Case 1:} $AA = e$.
\begin{longtable}{|c||c|c|c|c|c|c|}
\hline
$*$ & $e$ & $A$ & $B$ & $C$ & $D$ & $E$ \\ \hline\hline
$e$ & $e$ & $A$ & $B$ & $C$ & $D$ & $E$ \\ \hline
$A$ & $A$ & $e$ &     &     &     &     \\ \hline
$B$ & $B$ &     &     &     &     &     \\ \hline
$C$ & $C$ &     &     &     &     &     \\ \hline
$D$ & $D$ &     &     &     &     &     \\ \hline
$E$ & $E$ &     &     &     &     &     \\ \hline
\end{longtable}
$AB$ can be $C,D,E$.

\textsc{Case 1.1:} $AA = e, AB = C$.
\begin{longtable}{|c||c|c|c|c|c|c|}
\hline
$*$ & $e$ & $A$ & $B$ & $C$ & $D$ & $E$ \\ \hline\hline
$e$ & $e$ & $A$ & $B$ & $C$ & $D$ & $E$ \\ \hline
$A$ & $A$ & $e$ & $C$ &     &     &     \\ \hline
$B$ & $B$ &     &     &     &     &     \\ \hline
$C$ & $C$ &     &     &     &     &     \\ \hline
$D$ & $D$ &     &     &     &     &     \\ \hline
$E$ & $E$ &     &     &     &     &     \\ \hline
\end{longtable}
It seems that $AC$ can be $B,D$, or $E$ (3 choices).
But in fact
from $AB = C$, we get
\begin{align*}
           A(AB) &= AC \\
\THEREFORE (AA)B &= AC & & \text{by associativity axiom} \\
\THEREFORE eB    &= AC & & \text{since $AA = e$}\\
\THEREFORE B     &= AC & & \text{by neutrality axiom}
\end{align*}
Therefore there is only one choice for $AC$, i.e., $AC = B$.
Hence
\begin{longtable}{|c||c|c|c|c|c|c|}
\hline
$*$ & $e$ & $A$ & $B$ & $C$ & $D$ & $E$ \\ \hline\hline
$e$ & $e$ & $A$ & $B$ & $C$ & $D$ & $E$ \\ \hline
$A$ & $A$ & $e$ & $C$ & $B$ &     &     \\ \hline
$B$ & $B$ &     &     &     &     &     \\ \hline
$C$ & $C$ &     &     &     &     &     \\ \hline
$D$ & $D$ &     &     &     &     &     \\ \hline
$E$ & $E$ &     &     &     &     &     \\ \hline
\end{longtable}
(Pro tip: Don't just brute force and try all possible options.
Use the algebraic relations as much as possible to cut down on
the number of cases to examine. Also, keep the cases organized
so that you can iterate over all of them without missing any.)




%==============================================================================
\newpage
Q6.
Let $p$ be a prime.
Prove that $\sqrt{p}$ is irrational (i.e., not a fraction) using
the well-ordering principle.
Note: You must use the well-ordering principle.

(Hint: $\sqrt{2}$ is irrational is proven in discrete 1.
That was usually proved using proof by contradiction.
Redesign the proof to use WOP.
Then generalize the proof to and replace $2$ by $p$.)

\textsc{Solution}

\input{q6.tex}


%==============================================================================
\newpage
Q7.
Let $p, q$ be distinct primes.
Prove that if $p \mid a$ and $q \mid a$, then $pq \mid a$.
(This was used in the proof of RSA in class.)

You must only use fact in the notes.
\begin{enumerate}[nosep]
\item Definition of divisibility
\item Basic properties of divisibility such as $\pm 1 | a$ for all $a$
and linearity of divisibility.
\item Euclidean property
\item Euclid's lemma
\item Extended Euclidean property
\item Fundamental theorem of arithmetic
\end{enumerate}

\textsc{Solution}

\input{q7.tex}


\end{document}
