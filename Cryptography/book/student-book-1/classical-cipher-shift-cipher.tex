\section{Shift cipher}


\begin{defn} Shift Cipher \end{defn}

The \defterm{shift cipher} is a cipher where a message is shifted by a constant value. 

To unencrypt the message, the shift is applied to the encrypted message in reverse.

\begin{ex}
  Suppose we want to encrypt the string 'popcorn'.

  Let $s$ be the set of our alphabet, the lowercase English letters. Therefore $s = \{a, b, c, \dots, z\}$

  The shift cipher is not complex. It works by simply shifting the values of your alphabet by a constant integer.

  Let us use the shift of 7.

  So, we simply shift each letter of our plaintext message to the right by 7.

  Therefore, $\{p, o, p, c, o, r, n\}$
  becomes $\{w, v, w, j, v, y, u\}$

  What were to happen if we had used a shift of 7 on the word 'today'?

  This is where the shift cipher makes use of modular arithmetic. $t \to a$, because when you go past z, you simply wrap around back to the beginning of the alphabet, a.

  Now we know that our ciphertext 'wvwjvyu' is the encrypted version of 'popcorn'.

  To decrypt, one must simply apply the shift in the opposite direction to the ciphertext.

  Therefore, $\{w, v, w, j, v, y, u\}$ becomes $\{p, o, p, c, o, r, n\}$.
\end{ex}

\section{Encryption and Decryption}

The shift cipher is only a simple example of a type of cipher. It helps to discuss ciphers in a more general way.

We can describe a cipher as a pair of functions $E : P \to C$ and $D : C \to P$ where $E$ is called the encryption function and $D$ is the decryption function.

This allows us to consider $P$, the plaintext (unencrypted message) as follows:
\[
D(E(x)) = x \text{ for all } x \in P
\]

In plain words, this means you can decrypt the encrypted message to attain the original plaintext message. 

In the case of the Caesar Cipher, can express our encryption function as
\[
E(x) = (SHIFT + ASCII\_VALUE(x)) \% 26
\]

\section{Attacking the Shift Cipher}

Recall that the shift cipher is based off a simple shift value. In our
example, the length of the alphabet was 26. We represented our values
using 0 ... 25. Therefore, there were only 26 options for our shift value.

This means that we can bruteforce the shift values.

$0 \leq k \leq 25$. We can print all the decryptions of ciphertext $c$, by
simply decrypting by every possible shift value.

\textsc{Heuristic}

If the message is long enough, we can actually break the shift cipher
without needing to brute-force it. The letters in the English language
appear in fairly predictable frequencies. The letter \texttt{e} is
the most commonly appearing letter.

The cipher doesn't mutate at all. That is to say, the same $k$ is used
for every input. Therefore, if \texttt{e} $\to$ \texttt{i}, it will do so
for the whole plaintext. This means that if we determine a best guess for
which letter \texttt{e} $\to$ ?, we can guess at what the shift is. It is
simply the value of the ciphertext minus the value of \texttt{e} (mod 26).

Here is a table of the average probability of English letters a - z:

\begin{center}
  \begin{tabular} {|c|c|}
    \hline
    \textsc{Letter} & \textsc{Probability} \\
    \hline
    \texttt{e} & 0.127 \\
    \hline
    \texttt{t} & 0.091 \\
    \texttt{a} & 0.082 \\
    \texttt{o} & 0.075 \\
    \texttt{i} & 0.070 \\
    \texttt{n} & 0.067 \\
    \texttt{s} & 0.063 \\
    \texttt{h} & 0.061 \\
    \texttt{r} & 0.060 \\
    \hline
    \texttt{d} & 0.043 \\
    \texttt{l} & 0.040 \\
    \hline
    \texttt{c} & 0.028 \\
    \texttt{u} & 0.028 \\
    \texttt{m} & 0.024 \\
    \texttt{w} & 0.023 \\
    \texttt{f} & 0.022 \\
    \texttt{g} & 0.020 \\
    \texttt{y} & 0.020 \\
    \texttt{p} & 0.019 \\
    \texttt{b} & 0.015 \\
    \hline
    \texttt{v} & 0.010 \\
    \texttt{k} & 0.008 \\
    \texttt{j} & 0.002 \\
    \texttt{x} & 0.001 \\
    \texttt{q} & 0.001 \\
    \texttt{z} & 0.001 \\
    \hline
  \end{tabular}
\end{center}

These are merely the letters frequencies found in a large body of text.
They are sure to change in some degree depending on the sample text.

We can extend this ``commonly occurring logic can be extended to pairs of
letters, known as \defterm{digrams}. Why stop at pairs of two letters?
Of course, we can continue hunting for commonly occuring groups of letters.

