
(a)
To prove that $G''$ is a group, I will demonstrate $G''$ follows the four
group axioms (closure, associativity, identity, and neutrality) \\

Closure:

\begin{enumerate}[nosep]
\item[] Let $(g, h)$ and $(g',h') \in G \times G'$ \\
\item[] By the definition of $G \times G'$, we know that $g$ and $g' \in G$ and $h, h' \in G'$. \\
\item[] Since $G$ is a group, it is closed under operation $*$, and therefore $g * g' \in G$. Since $G'$ is a group, it is closed under the operation $*'$, and therefore $h *' h' \in G'$. Therefore, we have \\
\item[] $(g,h) *'' (g', h') = (g * g', h *' h') \in G \times G'$ \\
\item[] Thus, $G \times G'$ is closed under $*''$.
\end{enumerate}

Associativity: 
\begin{enumerate}[nosep]
\item[] Let $(g,h), (g',h'), (g'',h'') \in G \times G'$. Then
\item[] \begin{align*}
  ((g,h) *'' (g',h')) *'' (g'',h'') &=
  (g * g', h *' h') *'' (g'',h'') & &\text{by defn. of $*''$} \\
  &= ((g * g') * g'', (h *' h') *' h'') & &\text{by defn. of $*''$} \\
  &= (g * (g' * g''), h * (h' *' h'')) & &\text{by the assoc. axiom for $G \times G'$} \\
  &= (g, h) *'' ((g' * g''), (h *' h'')) & &\text{by defn. of $*''$} \\
  &= (g, h) *'' ((g', h'), (g'', h'')) & &\text{by defn. of $*''$} \\
\end{align*}
\item[] Therefore, $G \times G'$ is associative under $*''$.
\end{enumerate}

Identity:
\begin{enumerate}[nosep]
\item[] WTS: That $(e,e')$ is the identity of $G''$.
\item[] Proof:
\item[] Let $(g,h)$ be any element of $G''$. Then
  \begin{align*}
    (e,e') *'' (g,h) &= (e * g, e' *' h) & &\text{by the defn. of $*''$} \\
    &= (g,h) & &\text{by the identity axiom for G and H} \\
  \end{align*}
\item[] Therefore $(e, e')$ is the identity of $G''$. 
\end{enumerate}

Neutrality (Inverse axiom):
\begin{enumerate}[nosep]
\item[] WTS: That $(g^{-1},h^{-1})$ is the inverse of $(g,h) \in G \times G'$.
\item[] Proof:
\item[] Let $(g,h)$ be any element of $G''$ where $g$ has inverse $g^{-1} \in G$ and $h$ has inverse $h^{-1} \in G'$.
\item[] Then, applying the inverse properties of $G$ and $G'$ we have
\item[] \begin{align*}
  (g,h) *'' (g^{-1},h^{-1}) &= (g * g^{-1}, h * h^{-1}) & &\text{by the defn. of $*''$} \\
  &= (e, e') & &\text{by the inverses axiom of $G \times G'$} \\
\end{align*}
\item[] Similarly,
\item[] \begin{align*}
  (g^{-1},h^{-1}) *'' (g,h) &= (e, e') \\
\end{align*}
\item[] Hence, $(g^{-1}, h^{-1})$ is the inverse of $(g,h)$.
\end{enumerate}

I have now shown $G''$ to posess all four group axioms and hence, $G''$ is a group. \\

(b)
\begin{enumerate}[nosep]
\item[] Let $(g,h), (g',h') \in G \times G'$. Then
\item[] \begin{align*}
  (g,h)(g',h') &= ((g,g'),(h,h')) \\
  &= ((g',g)(h',h)) \\
\end{align*}
\item[] Since $G$ and $G'$ are abelian. But
\item[] \begin{align*}
  ((g',g)(h',h)) &= (g',h')(g,h) \\
\end{align*}
\item[] So the cartesian product $G \times G'$ is abelian.
\end{enumerate}

(c)
NEED TO DO THIS ONE STILL

(d)
\begin{enumerate}[nosep]
\item[] $\mathbb{Z}/2 \times \mathbb{Z}/3 = \{(0,0),(0,1),(0,2),(1,0),(1,1),(1,2)\}$ under addition $(+_2,+_3)$.
\item[] If we choose some element $x \in G = (1,1)$, the group can be written as $\{e,4a,2a,3a,a,5a\}$.
\item[] Thus, $\mathbb{Z}/2 \times \mathbb{Z}/3$ is generated by $x$ and therefore cyclic.
\item[] Being a cyclic group of order 6, we have $\mathbb{Z}/2 \times \mathbb{Z}/3 \cong \mathbb{Z}/6$.
\end{enumerate}

(e)
\begin{enumerate}[nosep]
\item[] $\mathbb{Z}/2 \times \mathbb{Z}/2 = \{(0,0),(1,0),(0,1),(1,1)\}$ under addition $(+_2,+_2)$. Suppose we have
\item[] $f:\mathbb{Z}/4 \to \mathbb{Z}/2 \times \mathbb{Z}/2$ as a homomorphism of additive groups.
\item[] This means that $\forall x \in \mathbb{Z}/2 \times \mathbb{Z}/2$ we have $x + x = 0$. Thus
\item[] $f(2) = f(1 + 1) = f(1) + f(1) = 0 = f(0)$
\item[] So $f$ is not injective, and therefore not an isomorphism.
\item[] So $\mathbb{Z}/4 \not\cong \mathbb{Z}/2 \times \mathbb{Z}/2$.
\end{enumerate}

