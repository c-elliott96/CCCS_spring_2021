\section{Affine cipher}

The Affine cipher encryption algorithm is as follows:
\[
E((a, b), x) = (ax + b) \hspace{0.5cm}\text{(mod 26)}
\]

The importance of this algorithm is the key value. Notice that it uses
two values, $a$ and $b$, in the form $ax + b$. This means that the
encryption value range is much larger, and there are many more keys
for an attacker to try to compute the plaintext value.

So, we know the formula for $E$ of the Affine cipher. What about $D$?

We can write $D$ in terms of $E$.
\begin{align*}
D((a, b), (E(a, b), x)) &= x \hspace{0.5cm}\text{(mod 26)}\\ 
D((a, b), ax + b) &= x \hspace{0.5cm}\text{(mod 26)}\\
\end{align*}
We know that every encryption must be decryptable for it to be used
as a cipher.
Let's assume that $D$ has the form
\[
D((a, b), x) = cx + d \hspace{0.5cm}\text{(mod 26)}
\]
If we apply some algebra to $E$ and $D$, we can arrive at
\[
D((a, b), ax + b) \equiv x \pmod{26}
\]
And then
\begin{align*}
  c(ax + b) + d &\equiv x \pmod{26} \\
  cax + cb + d &\equiv x \pmod{26} \\
\end{align*}
For now, let's assume
\[
cax \equiv x \pmod{26}
\]
and
\[
cb + d \equiv 0 \pmod{26}
\]
For the above to be true, c and d must meet some specific criteria.
Notice that $cax \equiv x \pmod{26}$. This means $c$ and $a$ must
somehow cancel out (in mod 26).

How do numbers cancel each other out in the modular world? Through
multiplicitive inverses! 

We will be able to discuss what to do with $cb + d \equiv 0 \pmod{26}$
once we look at $cax \equiv x \pmod{26}$ first.

We know that $c$ must be the multiplicitive inverse of $a \pmod{26}$.
The multiplicitive inverse of $a$, in this case denoted $c$, satisfies
\[
ca \equiv 1 \pmod{26}
\]
In other words, what number multiplied by $a$ equals 1 in mod 26?

The following python program produces all integers with multiplicitive
inverses in mod 26.
\begin{Verbatim}[frame=single]
for i in range(26):
  for j in range(26):
    if i * j % 26 == 1:
      print('%d is a mult. inverse of %d (mod 26)' % (j, i))
\end{Verbatim}
Outputs:
\begin{Verbatim}[frame=single]
[student@localhost tmp]$ python cipher-tools.py 
1 is a mult. inverse of 1 (mod 26)
9 is a mult. inverse of 3 (mod 26)
21 is a mult. inverse of 5 (mod 26)
15 is a mult. inverse of 7 (mod 26)
3 is a mult. inverse of 9 (mod 26)
19 is a mult. inverse of 11 (mod 26)
7 is a mult. inverse of 15 (mod 26)
23 is a mult. inverse of 17 (mod 26)
11 is a mult. inverse of 19 (mod 26)
5 is a mult. inverse of 21 (mod 26)
17 is a mult. inverse of 23 (mod 26)
25 is a mult. inverse of 25 (mod 26)
\end{Verbatim}
As you can see, not all integers in mod 26 have a multiplicitive
inverse. This means that the value chosen for $a$ in our cipher
must posess this quality.

Alright, so we know the value of $c$. This multiplicitive
inverse is denoted $a^{-1}$. Since we can now label $c$,
we can do more work on the second part of the equation:
\begin{align*}
  cb + d &\equiv 0 \pmod{26} \\
  a^{-1}b + d &\equiv 0 \pmod{26} \\
  d &\equiv -a^{-1}b \pmod{26} \\
\end{align*}
We then can put it all together:
\begin{align*}
  D((a, b), x) &\equiv a^{-1}x - a^{-1}b \pmod{26} \\
  &\equiv a^{-1}(x - b) \pmod{26} \\
\end{align*}
where $a$ must be invertible mod 26. 

\begin{ex}
  Encrypt the plaintext message \verb!gollum! using the key (3, 12) with the Affine Cipher.

  Let $x = \verb!gollum!$
  \begin{center}
      \begin{itemize}
      \item g $ = 6 \to 3(6) + 12 = 4 \pmod{26}$ \\
      \item o $= 14 \to 3(14) + 12 = 2 \pmod{26}$ \\
      \item l $= 11 \to 3(11) + 12 = 19 \pmod{26}$ \\
      \item l $= 11 \to 3(11) + 12 = 19 \pmod{26}$ \\
      \item u $= 20 \to 3(20) + 12 = 20 \pmod{26}$ \\
      \item m $= 12 \to 3(12) + 12 = 22 \pmod{26}$ \\
      \end{itemize}
  \end{center}
  Now we have $E((3, 12), x) = $\Verb![4, 2, 19, 19, 20, 22]! \\
  Converting the ciphertext integer values back to lowercase ascii
  we get
  \Verb!ecttuv!.
\end{ex}
