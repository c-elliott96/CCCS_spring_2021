\subsection{Fitting Lines -- Regression}
\label{sec:linefitting}
{
\pgfkeys{
	/pgfmanual/gray key prefixes={/pgfplots/table},
}

This section documents the attempts of \PGFPlots\ to fit lines to input coordinates. \PGFPlots\ currently supports |create col/linear regression| applied to columns of input tables. The feature relies on \PGFPlotstable, it is actually implemented as a table postprocessing method. 


\begin{stylekey}{/pgfplots/table/create col/linear regression=\marg{key-value-config}}%
\pgfkeys{
	/pgfmanual/gray key prefixes={/pgfplots/table/create col/linear regression/},
	/pdflinks/search key prefixes in/.add={/pgfplots/table/create col/linear regression/,}{},
}
	A style for use in |\addplot table| which computes a linear (least squares) regression $y(x) = a \cdot x + b$ using the sample data $(x_i,y_i)$ which has to be specified inside of \meta{key-value-config} (see below).

	It creates a new column on-the-fly which contains the values $y(x_i) = a \cdot x_i + b$. The values $a$ and $b$ will be stored (globally) into \declareandlabel{\pgfplotstableregressiona} and \declareandlabel{\pgfplotstableregressionb}.

	
\begin{codeexample}[]
\begin{tikzpicture}
	\begin{axis}[legend pos=outer north east]
	\addplot table[row sep=\\] {% plot X versus Y. This is original data.
		X Y\\
		1 1 \\
		2 4\\
		3 9\\
		4 16\\
		5 25\\
		6 36\\
	};
	\addplot table[row sep=\\,
		y={create col/linear regression={y=Y}}] % compute a linear regression from the input table
	{
		X Y\\
		1 1 \\
		2 4\\
		3 9\\
		4 16\\
		5 25\\
		6 36\\
	};
	%\xdef\slope{\pgfplotstableregressiona} %<-- might be handy occasionally
	\addlegendentry{$y(x)$}
	\addlegendentry{% 
		$\pgfmathprintnumber{\pgfplotstableregressiona} \cdot x  
		\pgfmathprintnumber[print sign]{\pgfplotstableregressionb}$}
	\end{axis}
\end{tikzpicture}
\end{codeexample}
	The example above has two plots: one showing the data and one containing the |linear regression| line. We use |y={create col/linear regression={}}| here, which means to create a new column\footnote{The \texttt{y=\{create col/} feature is available for any other \PGFPlotstable\ postprocessing style, see the \texttt{create on use} documentation in the \PGFPlotstable\ manual.} containing the regression values automatically. 
	As arguments, we need to provide the $y$ column name explicitly\footnote{In fact, \PGFPlots\ sees that there are only two columns and uses the second as default. But you need to provide it if there are at least 3 columns.}. The $x$ value is determined from context: |linear regression| is evaluated inside of |\addplot table|, so it uses the same $x$ as |\addplot table| (i.e.\ if you write |\addplot table[x=|\marg{col name}|]|, the regression will also use \marg{col name} as its |x| input). Furthermore, it shows the line parameters $a$ and $b$ in the legend.



	The following \meta{key-value-config} keys are accepted as comma--separated list:

	\begin{key}{%
		/pgfplots/table/create col/linear regression/table=\marg{\textbackslash macro {\normalfont or} file name} (initially empty)}
		Provides the table from where to load the |x| and |y| columns. It defaults to the currently processed one, i.e.\ to the value of |\pgfplotstablename|.
	\end{key}
	\begin{keylist}{%
		/pgfplots/table/create col/linear regression/x=\marg{column} (initially empty),
		/pgfplots/table/create col/linear regression/y=\marg{column} (initially empty)}
		Provides the source of $x_i$ and $y_i$ data, respectively. The argument \meta{column} is usually a column name of the input table, yet it can also contains |[index]|\meta{integer} to designate column indizes (starting with $0$), |create on use| specifications or |alias|es (see the \PGFPlotstable\ manual for details on |create on use| and |alias|).

		The initial configuration (an empty value) checks the context where the |linear regression| is evaluated. If it is evaluated inside of |\pgfplotstabletypeset|, it uses the first and second table columns. If it is evaluated inside of |\addplot table|, it uses the same $x$ input as the |\addplot table| statement. The |y| key needs to be provided explicitly (unless the table has only two columns).
	\end{keylist}

	\begin{keylist}{%
		/pgfplots/table/create col/linear regression/xmode=\mchoice{auto,linear,log} (initially auto),
		/pgfplots/table/create col/linear regression/ymode=\mchoice{auto,linear,log} (initially auto)}
		Enables or disables processing of logarithmic coordinates. Logarithmic processing means to apply $\ln$ before computing the regression line and $\exp$ afterwards.

		The choice |auto| checks if the column is evaluated inside of a \PGFPlots\ axis. If so, it uses the axis scaling of the embedding axis. Otherwise, it uses |linear|.

		In case of logarithmic coordinates, the |log basis x| and |log basis y| keys determine the basis.

\begin{codeexample}[]
\begin{tikzpicture}
\begin{loglogaxis}
 \addplot table[x=dof,y=error2] 
    {pgfplotstable.example1.dat};	
  \addlegendentry{$y(x)$}

 \addplot table[
  	x=dof,
  	y={create col/linear regression={y=error2}}] 
    {pgfplotstable.example1.dat};	

  % might be handy occasionally:
  %\xdef\slope{\pgfplotstableregressiona} 
 \addlegendentry{slope 
   $\pgfmathprintnumber{\pgfplotstableregressiona}$}
\end{loglogaxis}
\end{tikzpicture}
\end{codeexample}

	The (commented) line containing |\slope| is explained above; it allows to remember different regression slopes in our example.
	\end{keylist}

	\begin{keylist}{%
		/pgfplots/table/create col/linear regression/variance list=\marg{list} (initially empty),%
		/pgfplots/table/create col/linear regression/variance=\marg{column name} (initially empty)%
	}
	Both keys allow to provide uncertainties (variances) to single data points. 
	A high (relative) variance indicates an unreliable data point, a value of $1$ is standard.

	The |variance list| key allows to provide variances directly as comma--separated list, for example

	|variance list={1000,1000,500,200,1,1}|.

	The |variance| key allows to load values from a table \meta{column name}. Such a column name is (initially, see below) loaded from the same table where data points have been found. The \meta{column name} may also be a |create on use| name.
\begin{codeexample}[]
\begin{tikzpicture}
\begin{loglogaxis}
 \addplot table[x=dof,y=error2] 
	{pgfplotstable.example1.dat};	
  \addlegendentry{$y(x)$}

 \addplot table[
      x=dof,
  	  y={create col/linear regression={
        y=error2,
        variance list={1000,800,600,500,400}}
	  }
 ]
	{pgfplotstable.example1.dat};	

 \addlegendentry{slope 
  $\pgfmathprintnumber{\pgfplotstableregressiona}$}
\end{loglogaxis}
\end{tikzpicture}
\end{codeexample}

	If both, |variance list| and |variance| are given, the first one will be preferred. Note that it is not necessary to provide variances for every data point.
	\end{keylist}

	\begin{key}{/pgfplots/table/create col/linear regression/variance src=\marg{\textbackslash table {\normalfont or} file name} (initially empty)}
	Allows to load the |variance| from another table. The initial setting is empty. It is acceptable if the |variance| column in the external table has fewer entries than expected, in this case, only the first ones will be used.
	\end{key}
\end{stylekey}

\paragraph{Limitations:} Currently, \PGFPlots\ supports only linear regression, and it only supports regression together with |\addplot table|. Furthermore, long input tables might need quite some time. 
}
