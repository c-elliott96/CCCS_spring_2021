\section{Vigenere cipher}

The Vigenere cipher differs from the previous two ciphers in that it is
polyalphabetic.

Let's dig in.

You select a keyword. For example \lq\lq hello''. [7, 4, 11, 11, 14] \\
Suppose our plaintext is \verb!ihavethehighgroundnow!

Notice that the key is shorter than the plaintext message. No worries.
We simply must repeat the key as needed to satisfy the length of the
plaintext. (with some unused overflow)

Length(plaintext) = 21. Length of key = 5. We can repeat the key 5 times
to match the length of the plaintext.

The encryption works as follows: for every letter of the plaintext, we
shift the alphabet by the value of the corresponding letter in the key.
So, plaintext[0] = i = 8. key[0] = h = 7. Therefore, ciphertext[0]
= (8 + 7) \% 26 = 15. Therefore ciphertext[0] = 15 = p.

This process repeats for every value in the plaintext, shifting by the
value of the key[i] \% len(key). 

\begin{ex} Let's do a full example using the key and message we already discussed above.
  
  Let
  message = \verb!ihavethehighgroundnow!
  and 
  key = \verb!hello!

  Let's go ahead and convert both the key and the message to their numerical
  value.
  \begin{Verbatim}
    to_ints(message) = [8, 7, 0, 21, 4, 19, 7, 4, 7, 8, 6,
      7, 6, 17, 14, 20, 13, 3, 13, 14, 22]
  \end{Verbatim}
  and \Verb!to_ints(key) = [7, 4, 11, 11, 14]! \\
  Let's iterate through message and convert each value.
  \begin{Verbatim}
    ciphertext = []
    for i in range(len(message)):
      cipher_char = (message[i] + key[i % len(key)]) % 26
      ciphertext.append(cipher_char)
    return to_chrs(ciphertext)
  \end{Verbatim}
  This will give us
  \begin{Verbatim}
    [15, 11, 11, 6, 18, 0, 11, 15, 18, 22, 13,
      11, 17, 2, 2, 1, 17, 14, 24, 2, 3]
    = pllgsalpswnlrccbroycd
  \end{Verbatim}
\end{ex}

That is how we encrypt using the Vigenere cipher.
To decrypt, we must have the key. We can then simply take the ciphertext
and apply the key to it in the same way, only this time subtracting the value
of the key from the encrypted character. We will arrive at the decrypted char.

Todo: Attacking the Vigenere cipher using frequency analysis to determine the
key length. 
