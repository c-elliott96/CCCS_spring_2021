\documentclass[a4paper,
               headsepline,footsepline,
               openany,oneside,chapterprefix,
               12pt]
{scrartcl}

\usepackage{myassignment}

\renewcommand\AUTHOR{Dr.~Yihsiang Liow}
\renewcommand\SHORTAUTHOR{Dr.~Y.~Liow}

\renewcommand\COURSENAME{Cryptography and Computer Security}
\renewcommand\COURSESHORTNAME{Crypto \& Security}
\renewcommand\COURSENUMBER{CISS451}


\renewcommand\AUTHOR{John Doe}
\renewcommand\EMAIL{jdoe@gmail.com}
\renewcommand\TITLE{Assignment 4}

\renewcommand{\thethm}{\arabic{thm}}% Remove subsection from theorem counter representation

\begin{document}
\topmatter

\textsc{Objectives}
\begin{enumerate}[nosep]
\item Use group axioms
\item Write a complete proof for a fact regarding groups
\item Construct groups when given a size 
\end{enumerate}

Make sure you read the comments before Q1 very carefully
before hacking proofs.

\newpage
\textsc{Basic math writing}

If you have a sequence of deductions where the left-hand-side is the same,
you can write this:
\begin{align*}
x &= 1 + 2 + 3 + 4 + 5 \\
\THEREFORE x &= 3 + 3 + 4 + 5 \\
\THEREFORE x &= 6 + 4 + 5
\end{align*}

Of course this is a shorthand for saying
\lq\lq $x$ is 1 + 2 + 3 + 4 + 5.
Therefore $x$ is $3 + 3 + 4 + 5$.
Therefore $x$ is $6 + 4 + 5$."
It's \lq\lq proper math" (just like \lq\lq proper English") to write
\begin{align*}
x &= 1 + 2 + 3 + 4 + 5 \\
&= 3 + 3 + 4 + 5 \\
&= 6 + 4 + 5 
\end{align*}

Because that's the same as saying
\lq\lq $x$ is 1 + 2 + 3 + 4 + 5, \textit{which is} 3 + 3 + 4 + 5,
\textit{which is} 6 + 4 + 5."
When you need to change the left-hand side,
you would write
\begin{align*}
x &= 1 + 2 + 3 + 4 + 5 \\
&= 3 + 3 + 4 + 5 \\
&= 6 + 4 + 5 \\
\THEREFORE x + 1
&= 6 + 4 + 5 + 1\\
&= 10 + 5 + 1
\end{align*}

Or you can use words:
\begin{align*}
x &= 1 + 2 + 3 + 4 + 5 \\
&= 3 + 3 + 4 + 5 \\
&= 6 + 4 + 5
\end{align*}
\underline{and therefore}
\begin{align*}
x + 1
&= 6 + 4 + 5 + 1\\
&= 10 + 5 + 1
\end{align*}

When a sequence of deductions is not computational like the above,
it might be better not to write in the above form in a math paragraph
form, but rather in a text paragraph form:

\underline{Since $x$ is $6+4+5$, adding 1, we get $x + 1$ is $6+4+5+1$.}

This form of writing is best when you are not doing computations.
This is especially true when you have \lq\lq for all" and \lq\lq there exists"
quantifiers in your argument.
You can of course mix text paragraph form and math paragraph form.
For instance:

\underline{For all $x$ and $y$ in $\R$ with $x > 0$ and $y > 0$, there is some
integer $n$ such that}
\begin{align*}
           nx &> y \\
\THEREFORE n  &> y/x
\end{align*} 

Frequently you need to justify an argument.
For instance:

\underline{By the neutral axiom,}
\[
xy = (xy)e
\]
\underline{Therefore by the associativity axiom}
\[
xy = x(ye)
\]

Or you can do it this way:
\begin{align*}
xy &= (xy)e     & &\text{by the neutrality axiom}   \\
   &= x(ye)     & &\text{by the associativty axiom}
\end{align*}
(This style is for classroom use. Most books/research papers actually
prefer the previous style.)
Of course besides quoting axioms, you can also quote assumptions, theorems,
etc.
\begin{align*}
xy &= (xy)e     & &\text{by the neutrality axiom}   \\
   &= x(ye)     & &\text{by the associativty axiom} \\
   &= x(y(wz))  & &\text{by Theorem 42}
\end{align*}

Do not start your proof abruptly.
You for example reference a given assumption:

\underline{Since $x > 0$, we have}
\begin{align*}
           \sqrt{x}     &> 0 \\
\THEREFORE \sqrt{x} + 1 &> 1 \\
\THEREFORE \frac{\sqrt{x} + 1}{x} &> \frac{1}{x}
\end{align*}

Compare this to without the \lq\lq Since ...":
\begin{align*}
           \sqrt{x}     &> 0 \\
\THEREFORE \sqrt{x} + 1 &> 1 \\
\THEREFORE \frac{\sqrt{x} + 1}{x} &> \frac{1}{x}
\end{align*}

you'll see that the second version is very abrupt.

%==============================================================================
\newpage
\textsc{Writing proof by induction}

For proof by induction, state clearly your $P(n)$.
When you are proving the base case, say you are proving the base case.
When you are proving the inductive case, say you are proving the inductive
case. When you are done with the inductive case, say you are done.
When you are completely done, say so and make a recap. 
That's called good writing: lead the reader.
Read a proof by induction from any textbook (or on the web).
Here's a standard example.

Prove that $1 + 2 + \cdots + n = n(n+1)/2$.

We will prove the above by mathematical induction.
(That's called leading the reader.)
Let $P(n)$ be the statement:
\[
P(n): 1 + 2 + \cdots + n = n(n+1)/2
\]
for $n \geq 1$.

We will first prove the base case. When $n = 1$,
\[
1 + \cdots + n = 1
\]
and
\[
n(n+1)/2 = 1(1+1)/2 = 1
\]
Hence the base case $P(1)$ holds.

Next, we will prove the inductive case.
Assume that $P(n)$ holds, i.e.,
\[
1 + 2 + \cdots + n = \frac{n(n+1)}{2}
\]
Therefore we have the following:
\begin{align*}
1 + 2 + \cdots + n + (n+1) &= \frac{n(n+1)}{2} + (n + 1)           \\
                           &= (n+1) \left( \frac{n}{2} + 1 \right) \\
                           &= (n+1) \left( \frac{n + 2}{2} \right) \\
                           &= \frac{(n+1)((n + 1) + 1)}{2} 
\end{align*}
Therefore $P(n+1)$ holds and hence the inductive case holds.

Therefore by the principle of (weak) mathematical induction,
$P(n)$ holds for all $n \geq 1$, i.e.,
\[
1 + 2 + \cdots + n = \frac{n(n+1)}{2}
\]
holds for all $n \geq 1$.
\qed

Read the above proof carefully several times.
Then write your own version -- and compare it with the above.

When an argument is long, note how I state what I'm going to prove
and after the argument is complete, I 
remind the reader the goal.

Note also that in the above, there's one main argument.
But there are two sub-arguments: the base case and the inductive case
arguments.
That structure (just like the structure in a piece of code) has to be clear:    

\begin{enumerate}[nosep]
\item[] We will prove the above by mathematical induction. ...
\item[]
\item[] \hspace{1cm} We will first prove the base case. ...
\item[] \hspace{1cm} Hence the base case $P(1)$ holds.
\item[]
\item[] \hspace{1cm} Next, we will prove the inductive case. ...
\item[] \hspace{1cm} Therefore $P(n+1)$ holds and hence the inductive case holds.
\item[] 
\item[] Therefore by the principle of (weak) mathematical induction,
$P(n)$ holds for all $n \geq 1$, i.e.,
\[
1 + 2 + \cdots + n = \frac{n(n+1)}{2}
\]
holds for all $n \geq 1$.
\end{enumerate}

A well written proof has no holes (gaps), no errors, \textit{and reads
well}.
Think of you standing in front of a judge and jury defending your case:
you are the butlet but you did not do it.
The point is not whether \textit{you} believe yourself.
Of course you do.
But rather, can you make the \textit{judge and jury} believe you?

After you are done, read your proof again.
Even if there are no errors, can it be improved?
Can it be shortened?
Can it be written clearer?
Or can you reach the same conclusion by a different and better route?

%==============================================================================
\newpage\textsc{Quoting/referencing axioms etc}

Call the group axioms by these names:
\begin{enumerate}[nosep]
\li Closure axiom
\li Associativity axiom
\li Inverse axiom
\li Neutrality axiom
\end{enumerate}
We have three tiny propositions in the notes.
Call them
\begin{enumerate}[nosep]
\li Uniqueness of identity
\li Uniqueness of inverse
\li Cancellation property
\end{enumerate}
You can reference these in a text paragraph such as:

\underline{By the associativity axiom, from $x * (x * y) = z$, we get
$(x * x)* y = z$.}              

Or in a math paragraph:
\begin{align*}
           x * (x * y) &= z     \\
\THEREFORE (x * x)* y  &= z & & \text{by associativity axiom}
\end{align*}

%==============================================================================
\newpage
Q1. Let $f : G \rightarrow G'$ be an isomophism of group
$(G, *, e)$ and $(G', *', e')$. Prove the following
\begin{enumerate}[nosep]
\item[(a)] $f(e) = e'$
\item[(b)] If $x \in G$, then $f(x^{-1}) = f(x)^{-1}$. (Make sure you read
those two ${}^{-1}$ very carefully!)
\item[(c)] If $x \in G$, then $f(x^n) = f(x)^n$ for all integer $n$.
(Hint: First prove this is true for $n \geq 0$ -- use induction.)
\item[(d)] If $G$ is abelian, then $G'$ is also abelian.
\end{enumerate}

Besides (d), most of the algebraic facts of $G$ would correspond to the same
algebraic facts in $G'$ since they are isomorphic.
For instance if $G$ has exactly 10 elements of order 5, then $G'$
also has exactly 10 elements of order 5 and
if $G$ is a cyclic group of order 12,
then $G'$ is also a cyclic group of order 12, etc.  

(By the way, it should be clear that $f(f^{-1}(x)) = x$ and $f^{-1}(f(y)) = y$.
That has nothing to do with group theory.
That's just a fact about inverse functions.)

\SOLUTION


(a)

(b)

(c)

(d)


%==============================================================================
\newpage
Q2.
In the following, let $(G, *, e)$, $(G', *', e')$, and $(G'', *'', e'')$ be
groups.
\begin{enumerate}[nosep]

\item[(a)]
Let $\operatorname{id}_G : G \rightarrow G$ be the identity function
$\operatorname{id}(x) = x$. Prove that $\operatorname{id}$ is a group isomorphism.

\item[(b)] Let $f : G \rightarrow G'$ be a group isomorphism.
Since $f$ is a bijection, it has an inverse function
$f^{-1}: G' \rightarrow G$. Prove that $f^{-1}$ is also a group isomorphism.

\item[(c)] Suppose
$f : G \rightarrow G'$ and
$f': G' \rightarrow G''$
are group isomorphisms.
Prove that the composition of function $f' \circ f: G \rightarrow G''$ is
also a group isomorphism.

\end{enumerate}
The above says that the concept of group isomorphism is a an
equivalence relation on the collection of all groups: if $G, G', G''$ are groups,
then
\begin{enumerate}[nosep]
\li Reflexive: $G \simeq G$
\li Symmetric: $G \simeq G' \implies G' \simeq G$
\li Transitive: $G \simeq G', G' \simeq G'' \implies G \simeq G''$
\end{enumerate}

\SOLUTION


(a) Goal: Prove that id is a group isomorphism

\begin{enumerate}[nosep]
\item[] $\forall x \in G, id(x) = x$, thus id is reflexive by definition of id \\
\item[] $id_G(x) = x = id_G(x)$ thus, id is symmetric \\
\item[] $id_G(x) = x' = x$ so $id_G(x') = x'' = x$ and thus id is transitive \\
\end{enumerate}

(b)
\begin{enumerate}[nosep]
\item[] If $x,y \in G'$, then $ f(a)= x$, $f(b) = y$ for some $a,b \in G$
\item[] thus \begin{align*}
  f^{-1}(x * y) &= f^{-1}(f(a) * f(b)) \\
  &= f^{-1}(x) * f^{-1}(y) \\
  &= f^{-1}(x * y) \\
\end{align*}
\end{enumerate}

(c)
\begin{enumerate}[nosep]
\item[] Let $x \in G''$. Then $f'(a) = x$ for some $a \in G'$. 
\item[] Let $a \in G'$. Then $f(b) = a$ for some $b \in G$.
\item[] Since $a \in G'$ is attained through $b \in G$, and $x \in G''$ is attained through
  $a \in G'$, we have shown that function composition is a group isomorphism.
\end{enumerate}


%==============================================================================
\newpage
Q3. If $(G, *, e)$ and $(G', *', e')$ are groups, then you can define
$G'' = G \times G'$ (cartesian product)
as the set of tuples $(x, x')$ where $x \in G$ and
$x' \in G$, and you can also define $*''$ by
\[
(x,x') *'' (y,y') = (x * y, x' *' y') 
\]
Let $e'' = (e, e') \in G \times G'$.
\begin{enumerate}[nosep]
\item[(a)] Prove that $(G'', *'', e'')$ is a group.
(Make sure you state clearly what is the inverse
$(x, y)^{-1}$ of $(x, y)$ in $G \times G' = G''$.)
\item[(b)] Prove that if $G$ and $G'$ are abelian, then $G''$ is also abelian.
\item[(c)] Prove that $G \times \{e'\}$ is a subgroup of $G''$ which is
isomorphic to $G$. Make sure you state the isomorphism function and then
prove that it is a group isomorphism.
You can and should think of $G \times \{e'\}$ as an isomorphic copy of
$G$ inside $G''$.
(Note that the intersection of $G \times \{e'\}$ and $\{e\} \times G'$
is the simplest possible subgroup, i.e., $\{(e, e')\}$.)
\item[(d)] Is $\Z/6$ isomorphic to $\Z/2 \times \Z/3$? Why?
\item[(e)] Is $\Z/4$ isomorphic to $\Z/2 \times \Z/2$? Why?
\end{enumerate}
The above says that
\begin{enumerate}[nosep]
\li The cartesian product of groups is a group.
\li The product of abelian groups is abelian.
\li The product of $G, G'$ contains isomorphic copies of $G$ and $G'$.
\end{enumerate}
The concept
is somewhat similar to the fact that the $x$-- and $y$--axis (two $\R$'s)
appears inside (i.e., $\R^2$) and their intersection is $\{(0,0)\}$.
The group product allows you to create more groups.
Furthermore the group product can be easily understood if you
know everything about the individual component groups.
For instance since $\Z/2$ is a group (under $+$), you immediately
have the group $\Z/2 \times \Z/2$, $\Z/2 \times \Z/2 \times \Z/2$, etc.

\SOLUTION


(a)
To prove that $G''$ is a group, I will demonstrate $G''$ follows the four
group axioms (closure, associativity, identity, and neutrality) \\

Closure:

\begin{enumerate}[nosep]
\item[] Let $(g, h)$ and $(g',h') \in G \times G'$ \\
\item[] By the definition of $G \times G'$, we know that $g$ and $g' \in G$ and $h, h' \in G'$. \\
\item[] Since $G$ is a group, it is closed under operation $*$, and therefore $g * g' \in G$. Since $G'$ is a group, it is closed under the operation $*'$, and therefore $h *' h' \in G'$. Therefore, we have \\
\item[] $(g,h) *'' (g', h') = (g * g', h *' h') \in G \times G'$ \\
\item[] Thus, $G \times G'$ is closed under $*''$.
\end{enumerate}

Associativity: 
\begin{enumerate}[nosep]
\item[] Let $(g,h), (g',h'), (g'',h'') \in G \times G'$. Then
\item[] \begin{align*}
  ((g,h) *'' (g',h')) *'' (g'',h'') &=
  (g * g', h *' h') *'' (g'',h'') & &\text{by defn. of $*''$} \\
  &= ((g * g') * g'', (h *' h') *' h'') & &\text{by defn. of $*''$} \\
  &= (g * (g' * g''), h * (h' *' h'')) & &\text{by the assoc. axiom for $G \times G'$} \\
  &= (g, h) *'' ((g' * g''), (h *' h'')) & &\text{by defn. of $*''$} \\
  &= (g, h) *'' ((g', h'), (g'', h'')) & &\text{by defn. of $*''$} \\
\end{align*}
\item[] Therefore, $G \times G'$ is associative under $*''$.
\end{enumerate}

Identity:
\begin{enumerate}[nosep]
\item[] WTS: That $(e,e')$ is the identity of $G''$.
\item[] Proof:
\item[] Let $(g,h)$ be any element of $G''$. Then
  \begin{align*}
    (e,e') *'' (g,h) &= (e * g, e' *' h) & &\text{by the defn. of $*''$} \\
    &= (g,h) & &\text{by the identity axiom for G and H} \\
  \end{align*}
\item[] Therefore $(e, e')$ is the identity of $G''$. 
\end{enumerate}

Neutrality (Inverse axiom):
\begin{enumerate}[nosep]
\item[] WTS: That $(g^{-1},h^{-1})$ is the inverse of $(g,h) \in G \times G'$.
\item[] Proof:
\item[] Let $(g,h)$ be any element of $G''$ where $g$ has inverse $g^{-1} \in G$ and $h$ has inverse $h^{-1} \in G'$.
\item[] Then, applying the inverse properties of $G$ and $G'$ we have
\item[] \begin{align*}
  (g,h) *'' (g^{-1},h^{-1}) &= (g * g^{-1}, h * h^{-1}) & &\text{by the defn. of $*''$} \\
  &= (e, e') & &\text{by the inverses axiom of $G \times G'$} \\
\end{align*}
\item[] Similarly,
\item[] \begin{align*}
  (g^{-1},h^{-1}) *'' (g,h) &= (e, e') \\
\end{align*}
\item[] Hence, $(g^{-1}, h^{-1})$ is the inverse of $(g,h)$.
\end{enumerate}

I have now shown $G''$ to posess all four group axioms and hence, $G''$ is a group. \\

(b)
\begin{enumerate}[nosep]
\item[] Let $(g,h), (g',h') \in G \times G'$. Then
\item[] \begin{align*}
  (g,h)(g',h') &= ((g,g'),(h,h')) \\
  &= ((g',g)(h',h)) \\
\end{align*}
\item[] Since $G$ and $G'$ are abelian. But
\item[] \begin{align*}
  ((g',g)(h',h)) &= (g',h')(g,h) \\
\end{align*}
\item[] So the cartesian product $G \times G'$ is abelian.
\end{enumerate}

(c)
NEED TO DO THIS ONE STILL

(d)
\begin{enumerate}[nosep]
\item[] $\mathbb{Z}/2 \times \mathbb{Z}/3 = \{(0,0),(0,1),(0,2),(1,0),(1,1),(1,2)\}$ under addition $(+_2,+_3)$.
\item[] If we choose some element $x \in G = (1,1)$, the group can be written as $\{e,4a,2a,3a,a,5a\}$.
\item[] Thus, $\mathbb{Z}/2 \times \mathbb{Z}/3$ is generated by $x$ and therefore cyclic.
\item[] Being a cyclic group of order 6, we have $\mathbb{Z}/2 \times \mathbb{Z}/3 \cong \mathbb{Z}/6$.
\end{enumerate}

(e)
\begin{enumerate}[nosep]
\item[] $\mathbb{Z}/2 \times \mathbb{Z}/2 = \{(0,0),(1,0),(0,1),(1,1)\}$ under addition $(+_2,+_2)$. Suppose we have
\item[] $f:\mathbb{Z}/4 \to \mathbb{Z}/2 \times \mathbb{Z}/2$ as a homomorphism of additive groups.
\item[] This means that $\forall x \in \mathbb{Z}/2 \times \mathbb{Z}/2$ we have $x + x = 0$. Thus
\item[] $f(2) = f(1 + 1) = f(1) + f(1) = 0 = f(0)$
\item[] So $f$ is not injective, and therefore not an isomorphism.
\item[] So $\mathbb{Z}/4 \not\cong \mathbb{Z}/2 \times \mathbb{Z}/2$.
\end{enumerate}



%==============================================================================
\newpage
Q4.
Recall that $S_n$ is the symmetric group on $n$ symbols, i.e.,
$S_n$ is the set of bijections $\{1,...,n\}\rightarrow\{1,...,n\}$.
Each element of $S_n$ can be thought of as a permutation of $\{1,...,n\}$.
Recall that the bijection on $\{1, 2, 3, 4, 5, 6\}$
\begin{align*}
1 &\mapsto 3 \\
2 &\mapsto 1 \\
3 &\mapsto 2 \\
4 &\mapsto 6 \\
5 &\mapsto 5 \\
6 &\mapsto 4 
\end{align*}
can be written using permutation notation
\[
\sigma = 
\begin{pmatrix}
1 & 2 & 3 & 4 & 5 & 6 \\
3 & 1 & 2 & 6 & 5 & 4
\end{pmatrix}
\]
or cycle notation
\[
\sigma = (1 \ 3 \ 2) (4 \ 6) (5)
\]
The above is an element of $S_6$.
Note that it's clear that $|S_6| = 6!$.
Of course you can compose bijections os $S_n$ since they are functions.
For instance
suppose I pick $\sigma=(1)(2 \ 3 \ 4)$ and $\tau = (1 \ 2)(3 \ 4)$ in
$S_4$.
Then
\begin{align*}
(\tau \circ \sigma)(1) &= \tau(\sigma(1)) = \tau(1) = 2 \\
(\tau \circ \sigma)(2) &= \tau(\sigma(2)) = \tau(3) = 4 
\end{align*}
etc.~and of course $\tau \circ \sigma \in S_4$.
Define the identity function $\operatorname{id}_n$ of $S_n$ by
$\operatorname{id}_n(i) = i$ for all $i \in \{1,...,n\}$.
Then $(S_n, \circ, \operatorname{id}_n)$ is a group (this is easy to prove --
I'll just let you think about it.)

\begin{enumerate}[nosep]

\item[(a)]
Draw the group table of $S_3$. (Note that $|S_3| = 3! = 6$.)

\item[(b)]
Is $S_3$ abelian?

\item[(c)]
What is the size of the largest cyclic subgroup of $S_3$?
(For $\sigma \in S_3$, compute the order of $\sigma$, i.e., the smallest $k$ such that
$\sigma^k = (1)(2)(3)$. What is the largest possible $k$ for all $\sigma \in S_3$?).

\item[(d)] What is the size of the largest cyclic subgroup of $S_5$?

\item[(e)] Is $S_n$ abelian if $n > 3$? (I hope it's obvious that $S_1 = C_1 = \Z/1$ and $S_2 = C_2 = \Z/2$ and are
therefore dead easy.)

\end{enumerate}

\SOLUTION


\newpage
(a)
\begin{longtable}{|c||c|c|c|c|c|c|}
  \hline
$\circ$       & $(1)(2)(3)$ & $(1 \ 2)(3)$ & $(1 \ 3)(2)$ & $(1)(2 \ 3)$ & $(1 \ 2 \ 3)$ & $(1 \ 3 \ 2)$ \\ \hline \hline
$(1)(2)(3)$   &             &              &              &              &               &               \\ \hline 
$(1 \ 2)(3)$  &             &              &              &              &               &               \\ \hline
$(1 \ 3)(2)$  &             &              &              &              &               &               \\ \hline
$(1)(2 \ 3)$  &             &              &              &              &               &               \\ \hline
$(1 \ 2 \ 3)$ &             &              &              &              &               &               \\ \hline
$(1 \ 3 \ 2)$ &             &              &              &              &               &               \\ \hline
\end{longtable}

(b)

(c)

(d)

(e)


%==============================================================================
\newpage
Q5. Find all non-isomorphic groups of size 6.
First do a complete listing of all possible group tables.
Narrow it down so that isomorphic ones are identified.
For the remaining non-isomorphic groups, make them concrete
by recognizing them as the standard groups mentioned in the notes:
$\Z/N$, matrix groups, permutation/symmetric groups, etc.

\SOLUTION

(I'll do the first few steps for you.)

Let $G = \{e, A, B, C, D, E\}$ be a group of size 6. (Therefore
all the symbols $e, A, B, C, D, E$ are distinct.)
where $e$ is the identity element.
We have to complete the following group table:
\begin{longtable}{|c||c|c|c|c|c|c|}
\hline
$*$ & $e$ & $A$ & $B$ & $C$ & $D$ & $E$ \\ \hline\hline
$e$ &     &     &     &     &     &     \\ \hline
$A$ &     &     &     &     &     &     \\ \hline
$B$ &     &     &     &     &     &     \\ \hline
$C$ &     &     &     &     &     &     \\ \hline
$D$ &     &     &     &     &     &     \\ \hline
$E$ &     &     &     &     &     &     \\ \hline
\end{longtable}
Since $e$ is the identity element we have
\begin{longtable}{|c||c|c|c|c|c|c|}
\hline
$*$ & $e$ & $A$ & $B$ & $C$ & $D$ & $E$ \\ \hline\hline
$e$ & $e$ & $A$ & $B$ & $C$ & $D$ & $E$ \\ \hline
$A$ & $A$ &     &     &     &     &     \\ \hline
$B$ & $B$ &     &     &     &     &     \\ \hline
$C$ & $C$ &     &     &     &     &     \\ \hline
$D$ & $D$ &     &     &     &     &     \\ \hline
$E$ & $E$ &     &     &     &     &     \\ \hline
\end{longtable}
$AA$ can be $e, B, C, D$, or $E$.

\textsc{Case 1:} $AA = e$.
\begin{longtable}{|c||c|c|c|c|c|c|}
\hline
$*$ & $e$ & $A$ & $B$ & $C$ & $D$ & $E$ \\ \hline\hline
$e$ & $e$ & $A$ & $B$ & $C$ & $D$ & $E$ \\ \hline
$A$ & $A$ & $e$ &     &     &     &     \\ \hline
$B$ & $B$ &     &     &     &     &     \\ \hline
$C$ & $C$ &     &     &     &     &     \\ \hline
$D$ & $D$ &     &     &     &     &     \\ \hline
$E$ & $E$ &     &     &     &     &     \\ \hline
\end{longtable}
$AB$ can be $C,D,E$.

\textsc{Case 1.1:} $AA = e, AB = C$.
\begin{longtable}{|c||c|c|c|c|c|c|}
\hline
$*$ & $e$ & $A$ & $B$ & $C$ & $D$ & $E$ \\ \hline\hline
$e$ & $e$ & $A$ & $B$ & $C$ & $D$ & $E$ \\ \hline
$A$ & $A$ & $e$ & $C$ &     &     &     \\ \hline
$B$ & $B$ &     &     &     &     &     \\ \hline
$C$ & $C$ &     &     &     &     &     \\ \hline
$D$ & $D$ &     &     &     &     &     \\ \hline
$E$ & $E$ &     &     &     &     &     \\ \hline
\end{longtable}
It seems that $AC$ can be $B,D$, or $E$ (3 choices).
But in fact
from $AB = C$, we get
\begin{align*}
           A(AB) &= AC \\
\THEREFORE (AA)B &= AC & & \text{by associativity axiom} \\
\THEREFORE eB    &= AC & & \text{since $AA = e$}\\
\THEREFORE B     &= AC & & \text{by neutrality axiom}
\end{align*}
Therefore there is only one choice for $AC$, i.e., $AC = B$.
Hence
\begin{longtable}{|c||c|c|c|c|c|c|}
\hline
$*$ & $e$ & $A$ & $B$ & $C$ & $D$ & $E$ \\ \hline\hline
$e$ & $e$ & $A$ & $B$ & $C$ & $D$ & $E$ \\ \hline
$A$ & $A$ & $e$ & $C$ & $B$ &     &     \\ \hline
$B$ & $B$ &     &     &     &     &     \\ \hline
$C$ & $C$ &     &     &     &     &     \\ \hline
$D$ & $D$ &     &     &     &     &     \\ \hline
$E$ & $E$ &     &     &     &     &     \\ \hline
\end{longtable}
(Pro tip: Don't just brute force and try all possible options.
Use the algebraic relations as much as possible to cut down on
the number of cases to examine. Also, keep the cases organized
so that you can iterate over all of them without missing any.)




\end{document}
